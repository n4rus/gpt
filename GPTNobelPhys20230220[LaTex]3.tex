\documentclass[12pt]{article}
\usepackage{amsmath,amssymb,amsthm}
\usepackage[margin=1in]{geometry}
\usepackage{graphicx}
\usepackage[utf8]{inputenc}
\usepackage[english]{babel}
\usepackage[numbers]{natbib}
 
\title{\textbf{Using OpenAI to explore the Relationship between Physical Phenomena Using a Combination of Classical and Quantum Mechanics Principles}} 
\author{\textbf{Lucas E. J. Silva \& ChatGPT(OpenAI)}} 
 
\begin{document} 

    \maketitle

    \section*{Abstract}

    This paper presents an equation that describes the relationship between various physical phenomena, 
  including gravity, electric fields, magnetic fields, and temperature. The equation is derived from a combination 
  of classical and quantum mechanics principles. The equation is used to calculate the force of a system in terms 
  of its mass, distance, charge, velocity, and other parameters. The results of this equation can be used to better 
  understand the behavior of physical systems in various contexts. 

    \section*{Introduction}

    The study of physical phenomena has been a major focus of scientific research for centuries. 
  In particular, understanding the forces that govern the behavior of physical systems has been a major 
  area of interest. This paper presents an equation that describes the relationship between various physical 
  phenomena, including gravity, electric fields, magnetic fields, and temperature. This equation is derived 
  from a combination of classical and quantum mechanics principles and can be used to calculate the force of 
  a system in terms of its mass, distance, charge, velocity, and other parameters. 

    \section*{Equation Derivation} 
    The equation presented in this paper is derived from a combination of classical and quantum mechanics 
  principles. It takes into account gravity ($G$), electric fields ($E$), magnetic fields ($B$), temperature ($T$), 
  mass ($m_1m_2/r^2$), charge ($q$), velocity ($v$), current ($I$), amplitude ($A$), wave number ($k$), angular 
  frequency ($\omega$) , friction coefficient $\mu_0I/2\pi r$, acceleration due to gravity($a$) , Hubble 
  constant($H_0$) , speed of light($c$) , Planck's constant($\hbar t)$, Boltzmann constant($K)$, coefficients($C_n)$, 
  potentials($\phi_n(x))$. 
  The equation is as follows:
  \begin{multline*}
  F = G(m_1m_2/r^2) + qE + qv \times B + \mu_0I/2\pi r + Aexp[i(kx-\omega t)] + mF/a + 10^{15} G (2\pi/T)^{1/2} \\
  + 1.4M_{\odot} - (2\pi R^2B)/(3Ic^2) - H_0 \times (1.22 \times 10^8 m/s)^2 + (1.6 \times 10^{-34} m)^2 + 2.725 K \\
  - \sum_{n=1}^{n=N}\ C_n \phi_n(x)\ exp(-i(G(m_{1}m_{2}/r^2)+ \mu_{0}(H+M)+ qE+ qv \times B))/{\hbar t = 0}
   \end{multline*}

    \section*{Conclusion}

    This paper presented an equation that describes the relationship between various physical phenomena including 
  gravity, electric fields, magnetic fields and temperature. The equation was derived from a combination of classical 
  and quantum mechanics principles and can be used to calculate the force of a system in terms of its mass, distance 
  charge velocity and other parameters. The results obtained from this equation can be used to better understand the 
  behavior of physical systems in various contexts.

    \bibliographystyle{plainnat}   % use plainnat for bibliography style 
    \bibliography{references.bib}   % use references.bib for bibliography file name 

     \end{document}
